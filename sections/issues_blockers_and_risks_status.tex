%===============================================================================
\section{Issues, Blockers, and Risks Status}

% !!! Reference back to the project plan !!!

% Identify any issues or blockers that are currently affecting the project.
% Outline the steps being taken to mitigate these concerns.

% Identifies any problems or obstacles hindering the progress and potential risks, along with their resolution or mitigation plans.

A number of issues were identified, and in an attempt to minimise their impact on the project they were combined with a mitigation action. The issues and their corresponding mitigation plan can be seen in Table.\:\ref{tab:issues_and_mitigation}. 
\begin{table}[H]
    \centering
    \begin{tabularx}{\columnwidth}{|X|X|} \hline
         \textbf{Issue} & \textbf{Mitigation} \\ \hline
         1.2 \& 1.6: Compilation error in centralised-ai repository after merge & Allocate additional work force to solve the problem. \\ \hline
         1.5: nav2 stack error & Enlisting Pontus Svenssons help (the only other member with \ac{ros2} and nav2 experience). Additionally, asking the question on Stack Exchange to allow external resources to help find the answer. \\ \hline
         1.3.5: Difficulty implementing \ac{spi} & Reading from multiple sources online, reading manuals and data sheets. Looking at examples provided by Würth. Will look at the examples provided by STMicroelectronics, as well as other competitive teams such as TIGERs Mannheim and Delft Mercurians. \\ \hline
         2.1.3: Size constraints for parts & Redesigning and moving already made \ac{cad} models to make more room to fit components. \\ \hline
         1: Software modules have so far only been tested individually. There may be issues or even necessary functionality that is missing that is only apparent once integration has been attempted. & Integrate different software modules at the earliest possible moment, thus leaving time to fix any potential issues which might arise. \\ \hline
         1.1.2: Simulation is currently limited to only being able to run in real time. Without being able to control speed of simulation, the capacity to train the \ac{ai} model will be significantly reduced. & Dedicate time to try and find a way to control simulation speed. Worst case scenario, the loss of \ac{ai} performance has to be accepted. \\ \hline
         2: Components arriving late, this will delay development and testing of components. & The \textit{Hardware interface} will be implemented until the components arrive. \\ \hline
         2.2: No sponsor for \ac{pcb} manufacturing has been found. & The project will use an external \ac{pcb} supplier. \\ \hline
    \end{tabularx}
    \caption{All identified issues and the steps being taken to mitigate them.}
    \label{tab:issues_and_mitigation}
\end{table}


%===============================================================================